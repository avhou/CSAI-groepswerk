\documentclass[]{article}
\usepackage[backend=biber, style=numeric]{biblatex}
\usepackage{paralist}
\usepackage{hyperref}
\usepackage{graphicx}
\usepackage{booktabs}
\usepackage{subcaption}
\usepackage{wrapfig}
\graphicspath{ {./figures/} }
\usepackage{geometry}
\geometry{a4paper, portrait}

\addbibresource{report.bib}
\newtheorem{researchquestion}{RQ}

%opening
\title{Capita Selecta in Artificial Intelligence - Assignment 3}
\author{Arne Lescrauwaet \small(852617312) \and Joachim Verschelde \small(852594432) \and Alexander Van Hecke \small(852631385) \and Farid Rasoolzadeh Baghmisheh \small(TODO)}

\begin{document}

\maketitle

\section{Introduction}
This report details the step taken by Arne Lescrauwaet (852617312), Joachim Verschelde (852594432), Alexander Van Hecke (852631385) and Farid Rasoolzadeh Baghmisheh (TODO) for the assignment of the 2023/2024 Capita Selecta in Artificial Intelligence course organised by the Open University (\cite{ou}).

For this assignment we decided to expand on the experiments described in \cite{durability}.
In this paper, the authors investigate the use of an LLM model and a RAG setup \cite{rag} to asses company durability disclosures.
They investigate climate transition disclosures by examining the published sustainability reports of carbon-intensive companies.
A set of 64 indicators (questions) was gathered that investigated measures already taken or yet to be taken.
For each question, relevant passages in the published sustainability reports were retrieved and passed as additional context to the GPT-4 model.
Their main conclusions were :

\begin{itemize}
    \item Companies more often disclose information about measures yet to be taken (``Talk'' measures) than about measures that were already taken (``Walk'' measures).
    \item The RAG setup was judged by human experts and found to be mostly positive.
    \item An automated tool as developed by the authors can be a help to human experts, but are no replacements for human experts.
\end{itemize}

In this research, we want to expand on \cite{durability} in the following ways :

\begin{itemize}
    \item We examine data published by the fast fashion industry instead of carbon-intensive companies.
    \item We evaluate and compare several freely available LLM models instead of using one commercial LLM model.
    \item We evaluate additional indicators provided by dr. ir. Clara Maathuis of Open University \cite{ou}.
\end{itemize}

This research is related to the \textbf{text} cluster of the course \cite{ou}.
The text cluster introduces a number of techniques to improve the performance of LLM models for certain tasks.
One of the techniques discussed is RAG, first introduced in \cite{rag}.
RAG adds context when querying LLM models, allowing the model to get a better semantic understanding of the question and to generate a more relevant answer.
The main strenghts of a RAG based setup are \textbf{i)} the ability to access up-to-date information (no cutoff), \textbf{ii)} the ability to add domain specific knowledge, \textbf{iii)} efficiency, as the parameteric memory does not need to contain all knowledge and no retraining or fine-tuning is necessary, \textbf{iv)} the ability to generate more specific, more diverse and more factual answers and \textbf{v)} the scalability of the approach.
The main disadvantages of a RAG based setup are \textbf{i)} context limits of LLMs model limit how much additional information can be passed, \textbf{ii)} the performance of the retriever as the weakest link in the setup, \textbf{iii)} the additional latency and complexity introduced, \textbf{iv)} the need to keep the knowledge base up-to-date.
Evaluation of climate transition disclosures presents an excellent concrete use case of RAG.

\section{Goal}
discuss the objective of this project split it into multiple smaller objectives that will be tackled in this project
\begin{itemize}
    \item Main goal
    \item Smaller objectives and report outline
\end{itemize}

\section{Data analysis}
describe the dataset used considering entities such as features and target, and discuss processing and cleaning aspects like duplicates and outliers
\begin{itemize}
    \item Dataset description
    \item Data processing
    \item Data cleaning
    \item Data visualization and interpretation
\end{itemize}

\section{Methodology and Implementation}
describe the followed methodological approach and discuss important implementation choices providing concrete examples
\begin{itemize}
    \item Research Methodology discussion
    \item Design elaboration
    \item Implementation choices discussion
\end{itemize}

\section{Evaluation and Results}
discuss the evaluation method and criteria used, and interpret the results obtained considering both technical and ethical aspects (norms and values)
\begin{itemize}
    \item Evaluation mechanism discussion and evaluation criteria selected
    \item Results description using different data visualization methods.
    \item Interpretation of the results obtained and positioning in the field considering both relevant technical and ethical aspects
\end{itemize}

\section{Conclusions and Discussion}
summarize the approach followed in this project and draw conclusions based on the results obtained
\begin{itemize}
    \item Conclusions based on the approach followed and results obtained
    \item Discuss the meaning and findings of this project in a broader context considering the cluster selected and aim of this course
    \item Discuss limitations and future directions based on the findings obtained
\end{itemize}

\printbibliography

\end{document}